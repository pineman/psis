% !TEX encoding = UTF-8 Unicode
\documentclass[a4paper, titlepage, english]{article}
\usepackage[margin=2.5cm]{geometry}
%\usepackage{fontspec} % XeLaTeX
\usepackage[T1]{fontenc} % LaTeX
\usepackage[utf8]{inputenc} % LaTeX
\usepackage{newtxmath, newtxtext}
\usepackage{csquotes}
\usepackage{babel}
%\usepackage[backend=bibtex]{biblatex}
%\usepackage[backend=biber]{biblatex}
%\addbibresource{bibliography.bib}

\usepackage{indentfirst}
\usepackage{graphicx}
	\graphicspath{{images/}}
\usepackage{grffile}
\usepackage{float}
\usepackage[tableposition=top]{caption}
\usepackage{amsmath}
	%\allowdisplaybreaks
\usepackage[makeroom]{cancel}
\usepackage[arrowdel]{physics}
\usepackage{esint}
\usepackage{siunitx}
	\sisetup{inter-unit-product =\ensuremath{.}, output-complex-root=\ensuremath{j},
		complex-root-position=before-number}
\usepackage{hyperref}
\usepackage{listings}
	\lstset{basicstyle=\ttfamily, language=C, tabsize=4}
\usepackage{dirtree}

% Section styles
%\renewcommand{\thesection}{\Roman{section}}
%\renewcommand{\thesubsection}{\alph{subsection})}
%\renewcommand{\thesubsubsection}{\roman{subsubsection}.}
\renewcommand{\thesubsubsection}{\alph{subsubsection})}

% Useful commands
\newcommand{\eq}{\Leftrightarrow} % Equivalente
% Para numerar apenas uma equação
\newcommand\numberthis{\addtocounter{equation}{1}\tag{\theequation}}
\newcommand\e{\mathrm{e} }
\newcommand\jj{\mathrm{j} }

% Header and footer
\usepackage{fancyhdr}
\pagestyle{fancy}
\fancyhf{}
\lhead{Systems Programming}
\rhead{Project Assignment}
\lfoot{IST - MEEC}
\rfoot{Page \thepage}
\renewcommand{\headrulewidth}{1pt}
\renewcommand{\footrulewidth}{0.5pt}

% Document
\begin{document}
	\begin{titlepage}
		\center
		\textsc{\bfseries\LARGE Instituto Superior Técnico}\\[1cm] % Name of your university/college
		\includegraphics[height=1.5cm]{IST_Logo.pdf}\\[2.5cm]

		\textsc{\Large Electrical and Computer Engineering}\\[0.5cm] % Major heading such as course name
		\textsc{\Large Systems Programming}\\[0.5cm] % Minor heading such as course title
		\textsc{\large 2017/2018 2\textsuperscript{nd} Semester}\\[2cm]

		\rule{\textwidth}{1.6pt}\vspace*{-\baselineskip}\vspace*{2pt} % Thick horizontal line
		\rule{\textwidth}{0.4pt}\\[\baselineskip] % Thin horizontal line
			\textsc{\Huge Project Assignment}\\[0.2cm]
			\bigskip
			\textsc{\large Distributed Clipboard}\\[0.2cm]
		\rule{\textwidth}{0.4pt}\vspace*{-\baselineskip}\vspace{3.2pt} % Thin horizontal line
		\rule{\textwidth}{1.6pt}\\[6cm]

		\begin{minipage}{0.9\textwidth}
			\begin{flushleft} \large
				\begin{Large}\bfseries\textsc{Authors}\end{Large}\\[0.4cm]
				\begin{tabular}{l l l}
					André Agostinho	& 84001 & \normalsize andre.f.agostinho@tecnico.ulisboa.pt \\
					João Pinheiro		& 84086 & \normalsize joao.castro.pinheiro@tecnico.ulisboa.pt \\
				\end{tabular}
			\end{flushleft}
		\end{minipage}\\[0.5cm]

		%\large \textsc{Laboratório segunda-feira, 09h30-11h30, Grupo D\\
		%\large 12 de março de 2018}\\[1cm]
		\setcounter{page}{0}
	\end{titlepage}
	\tableofcontents \newpage

	\section{Introduction}
	\par
	The goal of this project was to implement a distributed clipboard and an application interface to it. Applications can copy (write) and paste (read) from and to regions on a local clipboard, which can be connected to another remote clipboard so as to synchronize their regions. The resulting network is a tree, as a clipboard can only connect to one other (the parent), but it can itself have connections to various children clipboards. There is always one clipboard which did not start by being connected, and that is the root of the tree. This tree is shown in <le figure ref>

	\section{Architecture}
	\subsection{Source Tree}
	\dirtree{%
	.1 /.
	.2 apps/.
	.3 copy.c // Hello.
	.3 paste.c.
	.3 paste.
	.3 copy.
	.2 cb\_common/.
	.3 cb\_common.h.
	.3 cb\_msg.h.
	.3 cb\_msg.c.
	.2 library/.
	.3 libclipboard.h.
	.3 libclipboard.c.
	.2 clipboard/.
	.3 app.h.
	.3 conn.h.
	.3 parent.h.
	.3 child.h.
	.3 child.c.
	.3 conn.c.
	.3 clipboard.
	.3 utils.c.
	.3 clipboard.h.
	.3 parent.c.
	.3 app.c.
	.3 utils.h.
	.3 region.h.
	.3 region.c.
	.3 main.c.
	}
	\subsection{Communication Protocol}
	\par
	To communicate between local applications and different clipboards, a simple protocol for messages was implemented as
	\begin{lstlisting}
uint8_t command;
uint8_t region;
uint32_t data_size;
uint8_t *data;
	\end{lstlisting}
	in which \texttt{data\_size} is the size of the data in the region that is sent immediately afterwards this header. This message header is marshalled into a buffer (of bytes) for sending into the network as

	\begin{table}[H]
	\centering
	\begin{tabular}{|*{6}{>{\centering\arraybackslash}p{.10\linewidth}|}}
		\hline
		0 & 1 & 2 & 3 & 4 & 5 \\ \hline
    \texttt{command} & \texttt{region} & \multicolumn{4}{c|}{\texttt{data\_size}} \\ \hline
    \multicolumn{6}{|c|}{\texttt{data} \ldots} \\ \hline
	\end{tabular}
	\end{table}
	\par
	All the functions related to communication are in \texttt{common/cb\_msg.c}, as they are shared by the library application interface and the clipboard server itself.

	\subsection{Connection Handling}
	\par
	Connections are handled in a one connection per thread model. There are three types of connections: local application connections, children clipboard connections and the optional connection to a parent clipboard. A connection, which is inexorably linked to a thread, is represented by a structure (in conn.h) as

	\begin{lstlisting}
struct conn {
	pthread_t tid;
	int sockfd;
	pthread_mutex_t mutex; // Protects writes to sockfd if needed
	struct conn *next;
	struct conn *prev;
};
	\end{lstlisting}
	in which the pointers \texttt{next} and \texttt{prev} are used to make a doubly-linked list of connections, of which there are two: local applications and children connections (in the implementation dummy head nodes are also used).

	\section{Threads}
	\subsection{Shared Data}
	\subsection{Synchronization}

	\section{Replication}

	\section{Error Handling}

	%\begin{figure}[h]
	%	\centering
	%	\includegraphics[width=0.6\linewidth]{figure.pdf}
	%	\caption{An example figure}
	%	\label{fig:figure-example}
	%\end{figure}

	%Ver \autoref{fig:figure-example}

	%\printbibliography
\end{document}
