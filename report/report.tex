% !TEX encoding = UTF-8 Unicode
\documentclass[a4paper, titlepage, english]{article}
\usepackage[margin=2.5cm]{geometry}
%\usepackage[no-math]{fontspec} % XeLaTeX
\usepackage{newtxtext, newtxmath}
\usepackage[T1]{fontenc} % LaTeX
\usepackage[utf8]{inputenc} % LaTeX
\usepackage[outputdir=build]{minted}
	\usemintedstyle{tango}
\usepackage{csquotes}
\usepackage{babel}
%\usepackage[backend=bibtex]{biblatex}
%\usepackage[backend=biber]{biblatex}
%\addbibresource{bibliography.bib}

\usepackage{indentfirst}
\usepackage{graphicx}
	\graphicspath{{images/}}
\usepackage{grffile}
\usepackage{float}
\usepackage[tableposition=top,skip=6pt]{caption}
\usepackage{amsmath}
	%\allowdisplaybreaks
\usepackage[makeroom]{cancel}
\usepackage[arrowdel]{physics}
\usepackage{esint}
\usepackage{siunitx}
	\sisetup{inter-unit-product =\ensuremath{.}, output-complex-root=\ensuremath{j},
		complex-root-position=before-number}
\usepackage{hyperref}

\usepackage{dirtree}
\usepackage{tabto}

% Section styles
%\renewcommand{\thesection}{\Roman{section}}
%\renewcommand{\thesubsection}{\alph{subsection})}
%\renewcommand{\thesubsubsection}{\roman{subsubsection}.}
%\renewcommand{\thesubsubsection}{\alph{subsubsection})}

% Useful commands
\newcommand{\eq}{\Leftrightarrow} % Equivalente
% Para numerar apenas uma equação
\newcommand\numberthis{\addtocounter{equation}{1}\tag{\theequation}}
\newcommand\e{\mathrm{e} }
\newcommand\jj{\mathrm{j} }

% Header and footer
\usepackage{fancyhdr}
\pagestyle{fancy}
\fancyhf{}
\lhead{Systems Programming}
\rhead{Project Assignment}
\lfoot{IST -- MEEC}
\rfoot{Page \thepage}
\renewcommand{\headrulewidth}{1pt}
\renewcommand{\footrulewidth}{0.5pt}

% Document
\begin{document}
\begin{titlepage}
	\center
	\textsc{\bfseries\LARGE Instituto Superior Técnico}\\[1cm] % Name of your university/college
	\includegraphics[height=1.5cm]{IST_Logo.pdf}\\[2.5cm]

	\textsc{\Large Electrical and Computer Engineering}\\[0.5cm] % Major heading such as course name
	\textsc{\Large Systems Programming}\\[0.5cm] % Minor heading such as course title
	\textsc{\large 2017/2018 2\textsuperscript{nd} Semester}\\[2cm]

	\rule{\textwidth}{1.6pt}\vspace*{-\baselineskip}\vspace*{2pt} % Thick horizontal line
	\rule{\textwidth}{0.4pt}\\[\baselineskip] % Thin horizontal line
	\textsc{\Huge Project Assignment}\\[0.2cm]
	\bigskip
	\textsc{\large Distributed Clipboard}\\[0.2cm]
	\rule{\textwidth}{0.4pt}\vspace*{-\baselineskip}\vspace{3.2pt} % Thin horizontal line
	\rule{\textwidth}{1.6pt}\\[6cm]

	\begin{minipage}{0.9\textwidth}
		\begin{flushleft} \large
			\begin{Large}\bfseries\textsc{Authors}\end{Large}\\[0.4cm]
			\begin{tabular}{l l l}
				André Agostinho & 84001 & \normalsize andre.f.agostinho@tecnico.ulisboa.pt    \\
				João Pinheiro   & 84086 & \normalsize joao.castro.pinheiro@tecnico.ulisboa.pt \\
			\end{tabular}
		\end{flushleft}
	\end{minipage}\\[0.5cm]

	\setcounter{page}{0}
\end{titlepage}
%\tableofcontents \newpage

\section{Introduction}
\par
The goal of this project was to implement a distributed clipboard and an application interface to it. Applications can copy (write) and paste (read) from and to regions on a local clipboard, which can be connected to another remote clipboard so as to synchronize their regions. The resulting network is a tree, as a clipboard can connect to only one other clipboard (the parent), but it can itself have connections to various children clipboards. There is always one clipboard which did not start by being connected, and that is the root of the tree. This tree is shown in <le figure ref>

\section{Architecture}
\subsection{Source Tree}
% TODO: replace by a figure?
\dirtree{%
	.1 /.
	.2 apps/.
	.3 copy.c.
	.3 paste.c.
	.3 wait.c.
	.3 copy.
	.3 paste.
	.3 wait.
	.2 cb\_common/.
	.3 cb\_common.h \tabto*{3cm} \# common macros between library and clipboard .
	.3 cb\_msg.h.
	.3 cb\_msg.c \tabto*{3cm} \# low level sending and receiving messages functions .
	.2 library/.
	.3 libclipboard.h \tabto*{3cm} \# apps include this header .
	.3 libclipboard.c \tabto*{3cm} \# library implementation (apps link to this .o) .
	.2 clipboard/.
	.3 clipboard \tabto*{3cm} \# clipboard binary .
	.3 main.c \tabto*{3cm} \# clipboard entry point .
	.3 clipboard.h \tabto*{3cm} \# global variables and error handling macros .
	.3 app.c \tabto*{3cm} \# application accept and handle functions/threads .
	.3 app.h.
	.3 child.c \tabto*{3cm} \# children accept and handle functions/threads .
	.3 child.h.
	.3 parent.c \tabto*{3cm} \# parent connect and handle functions/thread .
	.3 parent.h.
	.3 conn.c \tabto*{3cm} \# connection list functions (new, accept, cancel, etc.) .
	.3 conn.h.
	.3 region.c \tabto*{3cm} \# region updates and copy/paste logic .
	.3 region.h.
	.3 utils.c \tabto*{3cm} \# miscellaneous functions .
	.3 utils.h.
}
\subsection{Communication Protocol}
\par
To communicate between local applications and local clipboards and between remote clipboards, a simple format for messages was implemented as
\begin{minted}[tabsize=4]{c}
uint8_t command;
uint8_t region;
uint32_t data_size;
uint8_t *data;
\end{minted}
in which \texttt{data\_size} is the size of the data in the region that is sent immediately afterwards this header. This message header is marshaled into a buffer (of bytes) for sending into the network simply as shown in \autoref{tab:msgbuf}.
\begin{table}[ht]
	\centering
	\caption{Message marshaling into a buffer \texttt{msg}}
	\label{tab:msgbuf}
	\begin{tabular}{|*{6}{>{\centering\arraybackslash}p{.10\linewidth}|}}
		\hline
		\texttt{msg[0]}  & \texttt{msg[1]} & \texttt{msg[2]}                          & \texttt{msg[3]} & \texttt{msg[4]} & \texttt{msg[5]} \\ \hline
		\texttt{command} & \texttt{region} & \multicolumn{4}{c|}{\texttt{data\_size}}                                                       \\ \hline
		\multicolumn{6}{|c|}{\texttt{data} \ldots}                                                                                          \\ \hline
	\end{tabular}
\end{table}
\par
All the functions related to communication are in \texttt{common/cb\_msg.c}, as they are shared by the library application interface and the clipboard server itself.

\subsection{Connection Handling}
\par
Connections are handled in a one connection per thread model. There are three types of connections: local application connections, children clipboard connections and the optional connection to a parent clipboard if the clipboard was started in connected mode. A connection, which is inexorably linked to a thread, is represented by a structure (in \texttt{conn.h}), shown in Listing \ref{list:conn}, in which the pointers \texttt{next} and \texttt{prev} are used to make a doubly-linked list of connections. There are two connection lists: local applications and children connections (in the list implementation dummy head nodes are also used).
\begin{listing}[ht]
\begin{minted}[tabsize=4]{c}
struct conn {
	pthread_t tid;
	int sockfd;
	pthread_mutex_t mutex; // Protects writes to sockfd if needed
	struct conn *next;
	struct conn *prev;
};
\end{minted}
\caption{Connection structure}
\label{list:conn}
\end{listing}

\section{Threads}
\par
The clipboard creates several threads to handle many connections:
\begin{itemize}
	\item One thread blocking on \texttt{accept} on the local UNIX domain socket, waiting for local application connections
	\item One thread for each local application connection
	\item One thread blocking on \texttt{accept} on the remote INET socket, waiting for remote children clipboard connections
	\item One thread for each remote child clipboard connection
	\item One thread for the parent connection, if the clipboard was started in connected mode
	\item The main thread blocks most signals and creates the connection accept threads. Then it synchronously waits for signals and handles them (e.g., on \texttt{SIGINT}, the main thread exits and during its cleanup handler cancels and joins the rest of the active threads)
\end{itemize}

\subsection{Shared Data \& Synchronization}
\par
Global variables, declared in \texttt{clipboard.h}, are shared between threads. These variables are:
\begin{itemize}
	\item \mint{c}|pthread_t main_tid;|
	      The main thread id is global because it is used in the \texttt{cb\_eperror} macro to handle fatal errors (print error and exit). Since the clipboard is multi-threaded, and to clean up correctly, we must use \texttt{pthread\_cancel} instead of \texttt{exit}, and so many points in the program may need the main thread id.
	\item \mint{c}|pthread_t parent_serve_tid;|
	      The parent thread id is global because the parent connection might die during \texttt{copy\_to\_parent}, which means the parent thread must be canceled, and we chose to make the thread id global.
	\item \mint{c}|struct region regions[CB_NUM_REGIONS];|
	      The local regions, while only written/read by the parent thread, on \texttt{copy} commands, are also read by app threads during a \texttt{paste} command. Each region structure contains a \texttt{rwlock} to synchronize its use.
	\item \mint{c}|struct conn *child_conn_list;|
	\item \mint{c}|pthread_rwlock_t child_conn_list_rwlock; // Protects child_conn_list|
	      The list of children connections, and respective read-write lock, are shared between potentially all threads, during a \texttt{copy} command, to propagate updates down. If the clipboard is the root, app and children threads write to all children in \texttt{copy\_to\_children}, and, if it is not, the parent thread does. Also, when a new child connection is accepted in \texttt{child\_accept}, the list is updated.
	\item \mint{c}|struct conn *app_conn_list;|
	\item \mint{c}|pthread_rwlock_t app_conn_list_rwlock; // Protects app_conn_list|
	      The list of app connections, and respective read-write lock, while not used directly by other threads, it is used when canceling all ongoing connections on clipboard exit, during \texttt{cleanup\_app\_accept} (this is also true for the child connections, in the respective accept cleanup routine). Also, when a new app connection is accepted in \texttt{app\_accept}, the list is updated.
	\item \mint{c}|pthread_rwlock_t parent_conn_rwlock; // Protects parent_conn and root|
	\item \mint{c}|struct conn *parent_conn;|
	\item \mint{c}|bool root;|
	  The parent connection and the \texttt{root} are used by app and connection threads, during \texttt{on\_copy()} \texttt{(copy\-\_to\_parent)}, and as such must be protected by \texttt{parent\_conn\_rwlock}, to prevent race conditions when the parent thread is canceled.
	Although the \texttt{root} boolean is shared between threads, it is only potentially modified once by the parent thread. Since it's just an integer variable, no synchronization is needed.
\end{itemize}
\par
Additionally, all connection structures contain a mutex (as seen in Listing \ref{list:conn}) so that only one thread can write to a socket at a time, to prevent interleaving.

\section{Replication}
\par
In order to replicate regions across clipboard servers, we first note the network is in general a tree with a single root as in <le tree ref?>. When a copy request by an app or another child is received by the clipboard, it forwards the request to its parent, without copying it to its local regions. When the root clipboard receives a copy request, it updates its regions and sends it to its children. When a clipboard receives a copy from its parent, it also updates its regions and sends propagates it down to its children. In this way, updates are serialized by the root, taking advantage of the fact TCP guarantees ordered data transfer, and the whole network is not left in an incoherent state (excluding the propagation times).

\section{Error Handling}
\par
Some errors are fatal, while others are not. For example, any connection error during the \texttt{accept} loops, are printed but otherwise ignored, to keep the server accepting connections. What errors were considered to be fatal or not are easily recognizable in the source code by the use of the macro \texttt{cb\_eperror} (exit and perror) vs.\ the macro \texttt{cb\_perror}, which only prints the error. Additionally, fatal errors which demonstrate a completely unknown state are handled using the \texttt{assert} function.
\par
Errors during \texttt{read} or \texttt{write} calls are handled by canceling the corresponding connection thread. The cleanup handlers then free all the resources and delete the connection from the appropriate list. The result is that the clipboard server, on all tests performed, is \texttt{valgrind}-clean and exits gracefully.
\par
On exit, the main thread \texttt{cancel()}s and \texttt{join()}s the parent and the two \texttt{accept} threads. These two in turn \texttt{cancel()} all ongoing child and app threads. In addition, when an app, child or the parent connection dies, the corresponding thread is \texttt{detach()}ed and cleaned up. Although during the \texttt{copy} command (\texttt{on\_copy()} function) the parent and child connections are potentially written to and the connections may fail, they are never canceled there, as those connections will eventually return to their normal \texttt{recv()} loop, and will \texttt{detach()} themselves as explained above.

%\begin{figure}[h]
%	\centering
%	\includegraphics[width=0.6\linewidth]{figure.pdf}
%	\caption{An example figure}
%	\label{fig:figure-example}
%\end{figure}

%Ver \autoref{fig:figure-example}

%\printbibliography
\end{document}
